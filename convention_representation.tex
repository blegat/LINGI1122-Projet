%convention_representation : représentation informatique des données et résultats du programme
\paragraph{Conventions de représentation en entrée}
Posons $n$ le nombre de noeuds et $m$ le nombre d'arêtes et $u_i$ pour $i = 1, \ldots, n$ le $i$\ieme{} noeud.

Nous travaillons dans le cas général où le graphe n'est pas spécialement simple,
on ne peut donc représenter le graphe par une matrice d'adjacence, surtout que ça ne serait pas efficace pour un graphe sparse car la complexité spaciale d'une telle matrice est $\bigoh(n^2)$.

L'unique opération qu'on a besoin de faire sur le graphe est de demander la liste des arêtes partant d'un noeud.
Un graphe sous la forme d'une Adjacency List parait donc approprié car la complexité de cette opération pour un noeud $v$ serait alors $\bigoh(\deg(v))$ qui est imbatable car c'est aussi la taille de la sortie.

Si on a en entrée un Edge List, on peut passer en Adjacency List en $\bigoh(m)$ ce qui est acceptable car la complexité de l'algorithme est $\bigoh(m\log(n))$ mais pas vraimen désirable.

Demandons donc une Adjacency List \lstinline|Edge graph[][]|
où \lstinline|graph[i]| donne les arêtes partant du $i$\ieme{} noeud.
Représentons d'ailleurs les noeuds de façon neutre avec $u_i = i-1$.
On aura besoin pour chaque arête de connaitre le poids $w$ et le noeud d'arrivée $v$.
Seulement, pour rester dans la représentation générale des graphes,
définissons une arête comme suit
\begin{lstlisting}
class Edge {
  int u; % source
  int v; % destination
  int w; % weight
}
\end{lstlisting}
Le poids est ici également représenté par un \lstinline|int|.
On va considérer que la somme des poids des arêtes du graphe ne dépasse pas $2^{31}-1$.
On considère aussi qu'il y a moins de $2^{31}-1$ noeuds même si pour un graphe fort sparse, l'algorithme peut terminer assez rapidement même avec un graphe de $2^{32}$ noeuds.

Cette discussion peut être évitée en considérant ici \lstinline|int| comme une indication que ce sont des données qui peuvent être représentée de façon entière sans le lier spécifiquement au \lstinline|int| de Java.

Le noeud de départ et d'arrivée sont simplement spécifier en entier
également avec \lstinline|int u_0| pour le noeud de départ
et \lstinline|int u_1| pour le noeud d'arrivée.

\paragraph{Conventions de représentation en sortie}
Il y aura en sortie un unique entier représentant la longueure minimale que prend une chaîne de $u_0$ à $u_1$.

\paragraph{Conventions de représentation durant le programme}
Du au fait que $u_i = i-1$, $\ell(u_i)$ peut être trivialement représentée par un tableau d'entier \lstinline|int[] l| de taille $n$.
Une question se pose cependant sur la représentation de l'infini.
Comme toutes les longueurs sont positives, on peut utiliser une valeur sentinelle comme $-1$ sans poser d'ambiguïter.
On aurait donc
$$\lstinline|l[i]| =
\begin{cases}
  -1 & \text{si } \ell(u_{i+1}) = \infty\\
  \ell(u_{i+1}) & \text{sinon}.
\end{cases}$$

Vient maintenant la question de la structure de donnée nous permettant d'obtenir $\vmin$ dans l'algorithme~\ref{algo:dijkstra}.
Notre Adjacency List nous permet de faire la ligne 3 en $\bigoh(\degout(u') f(n))$ où $f(n)$ est la complexité de mettre à jour $\ell(v)$.
La ligne 4 a une complexité de $\bigoh(g(n))$ où $g(n)$ est la complexité de trouver $\vmin$.
La complexité de l'algorithme est donc $\bigoh(mf(n) + ng(n))$.
Comparons à présent plusieurs structures de donnée possibles nous permettant de rechercher $\vmin$
\begin{description}
  \item[Aucune] Avec simplement \lstinline|l|, on a $f(n) = \bigoh(1)$ et $g(n) = \bigoh(n)$ et donc $\bigoh(m + n^2)$ ce qui pour un graphe simple vaut $\bigoh(n^2)$ car $m \leq \frac{n(n-1)}{2}$ pour un graph simple. C'est une bonne complexité pour un graphe dense mais comme on va le voir on sait faire bien mieux pour un graphe un peu sparse.
  \item[Un heap avec location unaware entries] Comme on ne sait pas modifier la clef d'une entrée, lorsqu'on veut mettre à jour $\ell(v)$, il faut réajouter $v$ avec le nouveau $\ell(v)$ et se rappeler d'ignorer
    l'ancien quand il sortira de la priority queue. Ceci peut être fait en calculant $S$ à l'aide d'un Bitset.
    Dans le pire cas où on met à jour à chaque coup, le nombre d'entrée dans le heap peut atteindre $m$, ce qui donne
    $f(n, m) = \bigoh(\log(m))$ et $g(n, m) = \bigoh(\log(m))$ et pour un graphe simple,
    $f(n, m) = \bigoh(2\log(n))$ et $g(n, m) = \bigoh(2\log(n))$ ce qui fait aussi du $\bigoh(\log(n))$ mais avec une moins bonne constante.
    Ça rajoute aussi du $m \log(m)$ pour toutes les $m$ boucles inutiles dans le pire cas où on tirera une entrée à ignorer.
  \item[Un heap avec location aware entries] On peut alors modifier directement l'entrée correspondante à $v$ lorsqu'on met
    à jour $\ell(v)$ ce qui va simplement replacer $(v, \ell(v))$ dans l'arbre et qui est d'ailleurs bien plus léger qu'un ajout.
    On a alors simplement $f(n) = \log(n)$ et $g(n) = \log(n)$ ce qui fait une complexité $\bigoh(m\log(n))$ mais avec une constante bien
    inférieure à celle du précédent.
\end{description}
Le heap formant un arbre binaire complet, il peut être stocké de façon très efficace dans un tableau \lstinline|int[] heap|
où $\lstinline|l[heap[i]]| \leq \lstinline|l[heap[2*(i+1)-1]]|$ et $\lstinline|l[heap[i]]| \leq \lstinline|l[heap[2*(i+1)-2]]|$.
Pour la location awareness, il nous suffit d'un autre tableau \lstinline|int[] back|
donnant à chaque noeud sa place dans le tableau donc $\lstinline|back[v]| = \lstinline|i| \Leftrightarrow \lstinline|heap[i]| = \lstinline|v|$.
