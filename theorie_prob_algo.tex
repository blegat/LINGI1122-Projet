% theorie_prob_algo

Ayant remarqué quelques conditions nécessaires au bon fonctionnement de l'algorithme et les invariants utiles à sa construction, nous pouvons  présent décrire l'algorithme.
Pour cela, nous énonçons en premier lieu l'algorithme\footnote{On écrit ici la version de l'algorithme où on renvoit toutes les longueurs des plus courts chemins d'un noeud aux autres du graphe.
Pour ne trouver que la longueur du plus court chemin entre deux noeuds, il suffit d'arrêter l'itération lorsque $u'$ vaut le deuxième noeud de la paire formée avec $u_0$} \cite{cours_graphe}.
On introduit les notations suivantes : $v$ désigne un noeud du graphe $G$, $\ell(v)$ désigne la longueur courante, dans les itérations de l'algorithme, entre $v$ et le noeud $u_0$ fourni en argument et $w(u'v)$ désigne la longueur de l'arête de plus faible longueur joingant $u'$ et $w$, ce qui faut $\infty$ si $u'$ et $w$ ne sont pas adjacents  :

\begin{algorithm}[H]
 \KwData{Graphe $G$, $u_0$ un noeud de $G$}
 \KwResult{Longueurs des plus courts chemins entre un noeud $u_0$ et les autres noeuds du graphe}
 initialization : $\ell(u_0)=0$, $\ell(v)=\infty$ pour $v \neq u_0$, $S:=u_0$, $u'=u_0$ \;
 \While{$G\textbackslash S \neq \emptyset$}{
  pour chaque $v \notin S$, $\ell(v)=$min($\ell(v)$,$\ell(u')+w(u'v)$) \% \bsc{mise à jour de $\ell$} \;
	trouver $\vmin \notin S$ tel que $\ell(\vmin) \leq \ell(v)$ pour tout $v \notin S$\;
	$u':=\vmin$\;
	$S=S \cup \{ u' \}$\;
 }
 \caption{Algorithme de Dijkstra}
 \label{algo:dijkstra}
\end{algorithm}

\paragraph{}

Cet algorithme fonctionne donc par des mises à jour successives des longueurs des plus courts chemins liant un noeud $u_0$ aux autres noeuds du graphe.
Dans l'ensemble $S$ se trouvent les noeuds pour lesquels nous avons trouvé la longueur du plus court chemin le liant à $u_0$.
Nous remarquons donc que dans la phase d'initialisation, la longueur du plus court chemin entre $u_0$ et lui-même est mise à $0$ et que celles de $u_0$ aux autres noeuds du graphe est mise à $\infty$\footnote{Même s'il ne s'agit que d'un détail d'implémentation, notons que nous pourrions stocker les distances entre $u_0$ et les autres noeuds du graphe dans un tableau.}.
L'ensemble $S$ contient donc $u_0$, étant donné que comme il n'y a pas d'arête de poids négatif, la longueur du plus court chemin entre deux noeuds ne peut être inférieure à $0$.
Nous posons alors $u'=u_0$, avec $u'$ le noeud ajouté dans l'ensemble $S$ à l'itération précédente de l'algorithme.

\paragraph{}

Nous entrons alors dans une phase récursive de l'algorithme, qui se poursuit tant que tout les noeuds n'ont pas été ajoutés dans l'ensemble $S$, ce qui signifie que l'algorithme se poursuit tant que toutes les distances des plus courts chemins entre $u_0$ et les autres noeuds du graphe n'ont pas été trouvées.
Notons que dans le cas où nous souhaitons utiliser l'algorithme pour trouver la longueur du plus court chemin entre deux noeuds $u_0$ et $u_1$ et non pas entre un noeud $u_0$ et les autres noeuds du graphe, il suffit d'arrêter la récursion quand $u'=u_1$.

\paragraph{}

Tant que la condition d'itération est satisfaite, nous entrons donc dans la boucle.
La première étape est alors de mettre à jour les distances entre les noeuds hors de $S$ et $u_0$.
En effet, à la phase d'initialisation, nous avons posé les distances entre $u_0$ et tous les autres noeuds du graphe à $\infty$, mais la distance entre $u_0$ et ses voisins est plus petite que $\infty$.
Nous mettons donc à jour les longueurs par rapport à $u_0$ pour les noeuds hors de $S$ de la façon suivante : la longueur d'un chemin\footnote{Et non pas du plus court chemin, car le noeud $v$ n'est pas encore dans l'ensemble $S$, donc $\ell(v)$ n'est qu'une approximation de $d(u_0,v)$, cette dernière notation désignant la longueur du plus court chemin entre $u_0$ et $v$.} entre $u_0$ et $v$ est éale au minimum entre la longueur d'un chemin de $u_0$ à $v$, enregistrée à l'itération précédente, et la longueur du plus court chemin de $u_0$ à $u'$\footnote{Nous désignons cette fois-ci le plus court chemin car $u' \in S$.} additionnée par la longueur de l'arête de plus faible poids liant $u'$ à $v$.
Ainsi, au premier passage dans la boucle, les longueurs $\ell(v)$ seront mises à jour pour les noeuds voisins de $u_0$.
Remarquons d'or et déjà que l'algorithme ne remet à jour les distances que pour les noeuds pour lesquels il n'a pas encore trouvé la longueur du plus court chemin les liant à $u_0$, et non pas pour tous les noeuds du graphe.
Donc, au fur et à mesure des itérations, quand l'ensemble $S$ grandit, cette étape prend de moins en moins de temps, ce qui détermine en grande partie l'efficacité de l'algorithme.
En outre, cette étape est correcte : il n'est pas utile de reconsidérer les longueurs pour tous les noeuds du graphe étant donné qu'il n'y a pas d'arête de poids négatifs.
Nous expliciterons davantage cette propriété par la suite.

\paragraph{}

Une fois les distances mises à jour, nous trouvons alors le noeud hors de $S$ dont la longueur du chemin le liant à $u_0$ est plus petite ue pour tous les autres noeuds de $S$.
Nous avons alors trouvé la longueur du plus court chemin le liant à $u_0$, le rajoutons dans l'ensemble $S$ et le désignons par $u'$ dans la prochaine itération de l'algorithme.
Pourquoi est-il correct de rajouter ce noeud dans l'ensemble $S$ ? Nous allons expliquer ce résultat par une preuve pas récurrence.
Nous allons prouver par récurrence sur les itérations de l'algorithme que, une fois les distances mises à jour, $\ell(\vmin)$ désigne bien la longueur du plus court chemin liant $u_0$ à $\vmin$ et que pour $\ell(v)$, pour tout noeud $v$ du graphe, désigne la longueur du plus court chemin entre $u_0$ et $v$ en utilisant seulement des noeuds de $S$.

\paragraph{}

Considérons d'abord la première itération de l'algorithme (cas de base) : $\vmin$ désignera le voisin de $u_0$ relié à celui-ci par l'arête sortant de $u_0$ de plus faible poids, étant donné que l'étape de mise à jour des longueurs n'a d'effet que pour les noeuds voisins de $u_0$, la distance entre deux noeuds non-adjacents valant $\infty$.
Notons le poids de cette arête par $x$ positif ou nul et cette arête $A$.
La longueur du plus court chemin entre $u_0$ et $\vmin$ sera donc au plus $x$.
Comme toutes les autres arêtes liant $u_0$ à ses voisins sont de poids au moins équivalent et qu'il n'y a pas d'arêtes de poids négatifs, alors un autre chemin vers $\vmin$ ne peut qu'avoir une longueur supérieure ou égale à $x$.
En effet, supposons d'abord que toutes les autres arêtes reliées à $u_0$ soit de poids strictement supérieure à $x$.
Alors, en empruntant un chemin reliant $u_0$ à $\vmin$ en partant avec une autre arête que $A$, nous empruntons un poids strictement supérieure à $x$ et donc pas le plus court chemin car la longueur du plus court chemin reliant $u_0$  $\vmin$ est au plus de $x$\footnote{S'il y avait eu des poids négatifs dans le graphe, alors nous aurions pu ``rattraper'' ce poids supérieur à $x$ en empruntant une arête de poids négative par la suite.
Mais ceci est impossible étant donné que nous excluons les arêtes de poids négatif.}.
Supposons maintenant qu'il existe d'autres arêtes de poids $x$ entre $u_0$ et ses voisins.
Chacune de celles-ci conduit à un autre noeud $v_i$, relié à $\vmin$\footnote{Ces noeuds $v_i$ sont toujours reliés à $\vmin$ dans un graphe non-dirigé : il suffit d'emprunter l'arête dans l'autre sens pour revenir à $u_0$ et puis d'emprunter l'arête $A$.
Dans le cas d'un graphe dirigé, il peut ne pas exister de chemin entre $v_i$ et $\vmin$.}.
Etant donné que toutes les arêtes sont de poids positif ou nul, la longueur du plus court chemin entre $v_i$ et $\vmin$ est positive ou nulle.
Désignons la par $\ell_i$.
Ayant déjà emprunté une arête de poids $x$, si la $\ell_i > 0$, alors le plus court chemin liant $u_0$ à $\vmin$ sans emprunter l'arête $A$ en premier lieu est de poids strictement supérieur à $x$, ce qui n'est pas le plus court chemin.
Si $\ell_i = 0$, alors la longueur du plus court chemin liant $u_0$ à $\vmin$ sans emprunter l'arête $A$ en premier lieu est aussi de $x$.
Comme ceci est valable pour tous les $i$ désignant un voisin $v_i$ de $u_0$ relié à celui-ci par une arête de poids $x$, nous savons alors qu'il n'existe pas de chemin de chemin entre $u_0$ et $\vmin$ de poids inférieure à $x$.
Nous avons donc montré que $x$ est la longueur du plus court chemin reliant $\vmin$ à $u_0$ et donc ce chemin n'est pas unique : il suffit de considérer l'exemple où $u_0$ est relié à $\vmin$ par une arête de poids $x$ positif ou nul, à un autre noeud $v_2$ par une arête de poids $x$ et où $\vmin$ et $v_2$ sont réliés par une arête de poids nul.
Les plus chemins de $u_0$ à $\vmin$ sont donc : $u_0 - \vmin$ et $u_0 - v_2 - \vmin$, ceux-ci étant de poids égaux.

\paragraph{}

Enfin, montrons que $\ell(v)$, pour tout $v \in G$, désigne bien la longueur du plus court chemin entre $u_0$ et $v$ en utilisant seulement des noeuds de $S$, sachant qu'à la première itération, $S$ ne contient que $u_0$. La distance $\ell(u_0)$ n'est pas mise à jour et, valant 0, est bien égale à la longueur du plus court chemin de $u_0$ à $u_0$. Aussi, la longueur du plus court chemin, en utilisant simplement des noeuds de $S$, entre $u_0$ et un noeud $v$ adjacent à $u_0$ est bien égale à $w(u_0 v)$, le poids de l'arête de plus faible poids entre les deux noeuds. De plus, par l'étape de mise à jour de $\ell$ et le fait que $\ell(v)$ est initialisé à $\infty$ pour $v \neq u_0$, $\ell(v)$vaudra bien $w(u_0 v)$ après la mise à jour. Pour $v$ non-adjacent à $u_0$, la distance entre lui et $u_0$ en utilisant seulement des noeuds de $S$ vaut bien $\infty$. $\ell(v)$ pour $v$ non-adjacent n'est donc pas changé. 

\paragraph{}

Ayant démontré le cas de base, démontrons le cas récursif : le noeud $\vmin$ ajouté dans $S_n$\footnote{Nous désignons l'ensemble $S_n$ comme l'ensemble $S$ à la fin de l'itération $n$, donc après avoir exécuté $S:=S \cup \{u'\}$.} à l'itération $n+1$ est relié à $u_0$ par une longueur minimale $\ell(\vmin)$ et, à la fin de l'itération $n+1$, $\ell(v)$, pour tout $v \in G$, désigne la longueur du plus court chemin entre $u_0$ et $v$ en utilisant seulement des noeuds de $S_n$. Nous disposons pour cela de l'hypothèse de récurrence : $\ell(u')$ désigne la longueur du plus court chemin entre $u_0$ et $u'$ et $\ell(v)$, pour tout $v \in G$, défini à la fin de l'étape $n$,  désigne la longueur du plus court chemin entre $u_0$ et $v$ en utilisant seulement des noeuds de $S_n \textbackslash \{u'\}$.  

\paragraph{}

Montrons d'abord que $\ell(\vmin)$ désigne la longueur du plus court chemin entre $u_0$ et $\vmin$. Supposons par l'absurde qu'il existe une distance $\ell_x$ telle que $\ell_x < \ell(\vmin)$. Par l'hypothèse de récurrence, $\ell(\vmin)$ est la longueur du plus court chemin entre $u_0$ et $\vmin$ en utilisant simplement des noeuds de $S_n \textbackslash \{u'\}$. Donc, $\ell_x$ utilise des noeuds hors de $S$ pour aller jusque $\vmin$. Comme les arêtes sont de poids positifs ou nuls, si $\ell_x < \ell(\vmin)$, il existe un noeud $v_y \notin S$ tels que, pour au moins un $v_i$, $\ell(v_y) < \ell(v_i)$, en désignant par $v_i$ un noeud de $S$ emprunté par le chemin de $u_0$ à $\vmin$, de longueur $\ell(\vmin)$, n'empruntant que des noeuds de $S$. Or, ceci est impossible car si cela avait été le cas, nous aurions ajouté $v_y$ dans $S$ à une étape précédente de l'algorithme, par la ligne \textit{trouver $\vmin \notin S$ tel que $\ell(\vmin) \leq \ell(v)$ pour tout $v \notin S$}. Dès lors, $\ell(\vmin)$ désigne la longueur du plus court chemin entre $u_0$ et $\vmin$.

\paragraph{}

Pour ce qui est de la deuxième partie de la propriété, notons par $\ell_{n}(v)$ la longueur du plus court chemin entre $v$ et $u_0$ en utilisant seulement des noeuds de $S_n \textbackslash \{u'\}$, avec $u'$ le noeud ajouté à l'itération $n$ et par $\ell_{n+1}(v)$ la longueur du plus court chemin entre $v$ et $u_0$ en utilisant seulement des noeuds de $S_n$. Le seul noeud susceptible faire diminuer $\ell_{n}(v)$ est celui que nous ajoutons dans $S$, à savoir $u'$, par définition de $\ell_{n}(v)$. Or, la ligne \bsc{mise à jour e $\ell$} exploite justement ce nouveau noeud récemment ajouté pour mettre à jour $\ell_{n}(v)$ en $\ell_{n+1}(v)$, vu $l(u')$ est la longueur du plus court chemin entre $u_0$ et $u'$ et n'utilise de plus que des noeuds de $S_n$, par l'hypothèse de récurrence. Ceci démontre donc la propriété.

\paragraph{}

Au cours de la démonstration de l'algorithme, nous avons donc pu identifié les deux invariants suivants : 
\begin{itemize}
	\item pour $v \in S$, $\ell(v)=d(u_0,v)$, $d(u_0,v)$ désignant la longueur du plus court chemin entre $u_0$ et $v$. De plus, le plus court chemin de $u_0$ à $v$ reste dans $S$.
	\item pour $v \notin S$, $\ell(v) \geq d(u_0,v)$ et $\ell(v)$ est la longueur du plus court chemin de $u_0$ vers $v$ dont tous les noeuds internes sont dans $S$.
\end{itemize}

\paragraph{}

Signalons une fois encore que si le graphe avait comporté des arêtes négatives, les invariants n'auraient pas pu être satisfait. En effet, ce n'est pas parce qu'un noeud $v$ est dans $S$ que $\ell(v)$ désigne la longueur du plus court chemin de $u_0$ à $v$. En effet, considérons le graphe simple complet à trois noeuds $x$, $y$ et $z$. Soit l'arête $x-y$ de poids $3$, l'arête $x-z$ de poids $4$ et l'arête $y-z$ de poids $-2$. Soit $u_0=x$. A sa première itération, comme $3 < 4$, l'algorithme va ajouter $y$ dans $S$ et considérer $\ell(y)=3$. Ensuite, la mise à jour de $\ell(z)$, qui valait $4$ à la précédente itération, donne : $\ell(z)=min(\ell(z),\ell(y)+w(y,z))=min(4,3+(-2))=1$. L'algorithme s'arrêtera donc en posant $\ell(z)=1$, sans remettre à jour $\ell(y)$ à $2$. Sa solution est donc fausse. Pour arriver à un bon résultat, il devrait, à chaque étape, reconsidérer tous les chemins possibles dans S menant à chaque noeud, ce qui a une complexité exponentielle.
 
