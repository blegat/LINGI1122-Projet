%% Based on a TeXnicCenter-Template by Tino Weinkauf.
%%%%%%%%%%%%%%%%%%%%%%%%%%%%%%%%%%%%%%%%%%%%%%%%%%%%%%%%%%%%%

%%%%%%%%%%%%%%%%%%%%%%%%%%%%%%%%%%%%%%%%%%%%%%%%%%%%%%%%%%%%%
%% HEADER
%%%%%%%%%%%%%%%%%%%%%%%%%%%%%%%%%%%%%%%%%%%%%%%%%%%%%%%%%%%%%
\documentclass[a4paper,twoside,10pt]{report}
%En-t�te classique pour un rapport en EPL
%========================================
%\usepackage{fullpage} %si on veut en page compl�te

\usepackage[french]{babel} %En francais
\usepackage[utf8]{inputenc} %Caracteres accentu�s
\usepackage[T1]{fontenc} 	%Police accentu�e
\usepackage{lmodern}			%Police vectorielle (haute qualit�)
\usepackage{vmargin} %Marges normales en A4
\usepackage{amsmath} %Insertion d'equations
%\usepackage{theorem}
%\usepackage{ntheorem} 
 

\usepackage{amsfonts}
\usepackage{amssymb}
%\usepackage{algorithm}
\usepackage{arydshln}  % pour pointill�s dans matrice
\usepackage{algpseudocode}
%\usepackage{amsfont} % pour $\mathbb{R}$
\usepackage{amssymb}

\usepackage{graphicx} %Images dans le PDF
\usepackage{epstopdf} %importer des eps (� laisser sous package graphicx !
\graphicspath{{Figures/}}
\usepackage{float} 		%Flottants : Figures,Tables
\usepackage{color}		%Utilisation de couleurs
\definecolor{gris}{gray}{0.5}
\usepackage{eurosym}
\usepackage[hyphens]{url}
\usepackage{hyperref}
% \usepackage{breakurl}
\usepackage{wrapfig}
\usepackage{multirow} 
\usepackage{rotating}
\usepackage[small,bf]{caption}
\usepackage{textcomp}
\usepackage{subfigure}
\usepackage{amsthm}
\usepackage{pgf}
\usepackage{mathtools}
\usepackage[final]{pdfpages} 

\newtheorem*{theorem}{Th�or�me} % pour num�roter : \newtheorem{theorem}{Th�or�me}[section]
\newtheorem{mydef}{D�finition}

%========================================
\usepackage{listings}	%Inclusion de code-source
\lstset{							%Param�tres g�n�raux pour les inclusions de code
	flexiblecolumns=true,
	numbers=left,
	numberstyle=\ttfamily\tiny,
	keywordstyle=\textcolor{blue},
	stringstyle=\textcolor{red},
	commentstyle=\textcolor{green},
	breaklines=true,
	extendedchars=true,
	basicstyle=\ttfamily\scriptsize,
	showstringspaces=false,
	captionpos=b
	}
\renewcommand{\lstlistingname}{\textsc{Code}}	%Remplacer 'Listing' par 'Code' dans les l�gendes
\usepackage[french,boxed,linesnumbered]{algorithm2e}	%Package algorithme avec options


\hypersetup{
colorlinks,%
citecolor=black,%
filecolor=black,%
linkcolor=black,%
urlcolor=black
} 
\usepackage[framed,numbered,autolinebreaks,useliterate]{mcode}
%% Define a new 'leo' style for the package that will use a smaller font.
\makeatletter
\def\url@leostyle{%
  \@ifundefined{selectfont}{\def\UrlFont{\sf}}{\def\UrlFont{\small\ttfamily}}}
\makeatother
%% Now actually use the newly defined style.
\urlstyle{leo}


\makeatletter
\renewenvironment{thebibliography}[1]{%
%     \section*{\refname}%
%      \@mkboth{\MakeUppercase\refname}{\MakeUppercase\refname}%
      \list{\@biblabel{\@arabic\c@enumiv}}%
           {\settowidth\labelwidth{\@biblabel{#1}}%
            \leftmargin\labelwidth
            \advance\leftmargin\labelsep
            \@openbib@code
            \usecounter{enumiv}%
            \let\p@enumiv\@empty
            \renewcommand\theenumiv{\@arabic\c@enumiv}}%
      \sloppy
      \clubpenalty4000
      \@clubpenalty \clubpenalty
      \widowpenalty4000%
      \sfcode`\.\@m}
     {\def\@noitemerr
       {\@latex@warning{Empty `thebibliography' environment}}%
      \endlist}
\makeatother




\makeatletter
\def\clap#1{\hbox to 0pt{\hss #1\hss}}%
\def\ligne#1{%
\hbox to \hsize{%
\vbox{\centering #1}}}%
\def\haut#1#2#3{%
\hbox to \hsize{%
\rlap{\vtop{\raggedright #1}}%
\hss
\clap{\vtop{\centering #2}}%
\hss
\llap{\vtop{\raggedleft #3}}}}%
\def\bas#1#2#3{%
\hbox to \hsize{%
\rlap{\vbox{\raggedright #1}}%
\hss
\clap{\vbox{\centering #2}}%
\hss
\llap{\vbox{\raggedleft #3}}}}%
\def\maketitle{%
\thispagestyle{empty}\vbox to \vsize{%
\ligne{\@blurbb}
\vspace{1cm}
\haut{}{\large \@blurb}{}
\vspace{1cm}
\haut{}{\large \bf \@codeCours : \@nomCours}{}
\vspace{1cm}
\haut{}{\bf \@nomDoc}{}
\vfill
\vspace{1cm}
\begin{flushleft}
\usefont{OT1}{ptm}{m}{n}
\huge \@title
\end{flushleft}
\par
\hrule height 2pt %la grosse ligne
\par
\begin{flushright}
\usefont{OT1}{phv}{m}{n}
\@author
\par
\end{flushright}
\vspace{1cm}
\vfill
\vfill
\bas{}{\large \@prof}{}
\vspace{1cm}
\bas{}{\large \@date}{}
}%
\cleardoublepage
}
\def\date#1{\def\@date{#1}}
\def\author#1{\def\@author{#1}}
\def\title#1{\def\@title{#1}}
\def\location#1{\def\@location{#1}}
\def\blurb#1{\def\@blurb{#1}}
\def\blurbb#1{\def\@blurbb{#1}}
\def\nomCours#1{\def\@nomCours{#1}}
\def\codeCours#1{\def\@codeCours{#1}}
\def\nomDoc#1{\def\@nomDoc{#1}}
\def\prof#1{\def\@prof{#1}}
\date{\today}
\author{}
\title{}
\blurb{}
\makeatother
\title{L'algorithme de Dijkstra}
\author{Group 8 :\\
Cyril \bsc{de Bodt}\\
Alexandre \bsc{Laterre}\\
Benoit \bsc{Legat}\\
Alexis \bsc{Pierret}\\
F\'elicien \bsc{Schiltz}}
\date{Février 2014}
\blurb{%
Université Catholique de Louvain\\
Ecole Polytechnique de Louvain\\
}% 

\codeCours{
LINGI1122}
\nomCours{
Méthodes de conception de programmes
}%
\nomDoc{%
}%
\blurbb{%
\begin{tabular}{ccc}
\includegraphics[height=3cm]{Images/logoucl}
&
\hspace{5cm}
&
\includegraphics[height=3cm]{Images/logoepl}
\end{tabular} 
}
\prof{%
Professeur : José \bsc{Vander Meulen} 
}%


% Alternative Options:
%	Paper Size: a4paper / a5paper / b5paper / letterpaper / legalpaper / executivepaper
% Duplex: oneside / twoside
% Base Font Size: 10pt / 11pt / 12pt


%% Language %%%%%%%%%%%%%%%%%%%%%%%%%%%%%%%%%%%%%%%%%%%%%%%%%
\usepackage[T1]{fontenc}

\usepackage{lmodern} %Type1-font for non-english texts and characters



%% Packages for Graphics & Figures %%%%%%%%%%%%%%%%%%%%%%%%%%
\usepackage{graphicx} %%For loading graphic files
%\usepackage{subfig} %%Subfigures inside a figure
%\usepackage{tikz} %%Generate vector graphics from within LaTeX

%% Please note:
%% Images can be included using \includegraphics{filename}
%% resp. using the dialog in the Insert menu.
%% 
%% The mode "LaTeX => PDF" allows the following formats:
%%   .jpg  .png  .pdf  .mps
%% 
%% The modes "LaTeX => DVI", "LaTeX => PS" und "LaTeX => PS => PDF"
%% allow the following formats:
%%   .eps  .ps  .bmp  .pict  .pntg


%% Math Packages %%%%%%%%%%%%%%%%%%%%%%%%%%%%%%%%%%%%%%%%%%%%
\usepackage{amsmath}
\usepackage{amsthm}
\usepackage{amsfonts}


%% Line Spacing %%%%%%%%%%%%%%%%%%%%%%%%%%%%%%%%%%%%%%%%%%%%%
%\usepackage{setspace}
%\singlespacing        %% 1-spacing (default)
%\onehalfspacing       %% 1,5-spacing
%\doublespacing        %% 2-spacing


%% Other Packages %%%%%%%%%%%%%%%%%%%%%%%%%%%%%%%%%%%%%%%%%%%
%\usepackage{a4wide} %%Smaller margins = more text per page.
%\usepackage{fancyhdr} %%Fancy headings
%\usepackage{longtable} %%For tables, that exceed one page
%\usepackage{color}

%%%%%%%%%%%%%%%%%%%%%%%%%%%%%%%%%%%%%%%%%%%%%%%%%%%%%%%%%%%%%
%% Remarks
%%%%%%%%%%%%%%%%%%%%%%%%%%%%%%%%%%%%%%%%%%%%%%%%%%%%%%%%%%%%%
%
% TODO:
% 1. Edit the used packages and their options (see above).
% 2. If you want, add a BibTeX-File to the project
%    (e.g., 'literature.bib').
% 3. Happy TeXing!
%
%%%%%%%%%%%%%%%%%%%%%%%%%%%%%%%%%%%%%%%%%%%%%%%%%%%%%%%%%%%%%

%%%%%%%%%%%%%%%%%%%%%%%%%%%%%%%%%%%%%%%%%%%%%%%%%%%%%%%%%%%%%
%% Options / Modifications
%%%%%%%%%%%%%%%%%%%%%%%%%%%%%%%%%%%%%%%%%%%%%%%%%%%%%%%%%%%%%

%\input{options} %You need a file 'options.tex' for this
%% ==> TeXnicCenter supplies some possible option files
%% ==> with its templates (File | New from Template...).

\newcommand\bigoh{\mathcal{O}}
\newcommand\degout{\deg_\mathrm{out}}
\newcommand\vmin{v_\mathrm{min}}

%%%%%%%%%%%%%%%%%%%%%%%%%%%%%%%%%%%%%%%%%%%%%%%%%%%%%%%%%%%%%
%% DOCUMENT
%%%%%%%%%%%%%%%%%%%%%%%%%%%%%%%%%%%%%%%%%%%%%%%%%%%%%%%%%%%%%
\begin{document}

\pagestyle{empty} %No headings for the first pages.


%% Title Page %%%%%%%%%%%%%%%%%%%%%%%%%%%%%%%%%%%%%%%%%%%%%%%
%% ==> Write your text here or include other files.

%% The simple version:
%\title{Title of this document}
%\author{Firstname Lastname}
%\date{} %%If commented, the current date is used.
\maketitle
%\newpage

%% The nice version:
%\input{titlepage} %%You need a file 'titlepage.tex' for this.
%% ==> TeXnicCenter supplies a possible titlepage file
%% ==> with its templates (File | New from Template...).


%% Inhaltsverzeichnis %%%%%%%%%%%%%%%%%%%%%%%%%%%%%%%%%%%%%%%
\tableofcontents %Table of contents
\cleardoublepage %The first chapter should start on an odd page.

\pagestyle{plain} %Now display headings: headings / fancy / ...


\chapter*{Introduction}
% introduction

Dans le cadre du cours de \texttt{Méthodes et conceptions de programmes}, il nous a été proposé d'étudier le problème de l'algorithme de Dijkstra. Cet algorithme revêt une importance considérable en théorie des graphes, un domaine qui prend de plus en plus d'ampleur de nos jours. Afin d'introduire la notion de graphe et l'algorithme de Dijkstra, considérons un \bsc{gps} dont la charge est de déterminer la plus court chemin entre deux villes fournies. Nous pouvons modéliser ce problème en représentant les villes comme les noeuds d'un graphe et les arêtes entre les noeuds comme les routes entre les villes. Chacune des arêtes sera pondérée par un nombre naturel\footnote{Dans le cadre de l'algorithme de Dijkstra, nous verrons qu'il est requis de n'avoir que des poids positifs ou nuls sur les arêtes.}. Le poids d'une arête peut donc exprimer l'avantage que l'on a à empruter la route correspondante (les autoroutes se verront attribuées des poids plus petit que les petites routes de campagne, dans le cadre de la recherche du plus court chemin entre deux villes). Ainsi, un chemin d'une ville à l'autre revient à emprunter une suite d'arêtes, chacune reliée à la précédente par un noeud. La longueur d'une telle suite d'arêtes est alors définie comme la somme des poids des arêtes de la suite.

\paragraph{}
Dans le cadre de notre exemple ci-dessus, il s'agit de trouver la plus faible longueur d'un chemin entre deux noeuds du graphe. Ce problème est celui résolu par l'algorithme de Dijkstra, dans le cas où tous les poids des arêtes sont positifs. Dans la suite du document, nous tenterons de spécifier davantage la théorie du problème ainsi que les conventions de représentation que nous nous sommes proposés d'utiliser.
\chapter{Théorie du problème}
% theorie probleme
\section{Enoncer le problème}
%enoncer_probleme

Après avoir introduit notre problème au cours de l'introduction, énonçons le de manière plus formelle : il s'agit de trouver, dans un graphe pondéré donné, la longueur du plus court chemin entre deux noeuds d'une paire donnée, ou bien les longueurs des plus courts chemins entre un noeud donné et tous les autres noeuds du graphe. Nous verrons en section \ref{sec:poser_prob} qu'il est nécessaire que les poids sur les arêtes soient positifs ou nuls. Le programme doit donc retourner un entier représentant la longueur du plus court chemin entre deux noeuds donnés dans un graphe donné ou une structure de donnée contenant les longueurs des plus courts chemins entre un noeud donné dans un graphe donné et les autres noeuds du graphe. La structure de donnée employée sera spécifiée en section \ref{subsec:convention_representation}.
\section{Poser le problème : analyse des besoins}
\label{sec:poser_prob}
% poser_prob
Ayant énoncer notre problème, commençons dès à présent à l'étudier. Quelles sont les ``conditions de fonctionnement de l'algorithme'' ? A première vue, les conditions suivantes semblent évidentes :
\begin{itemize}
	\item Les noeuds fournis en argument du programme doivent appartenir au même graphe, la distance entre deux noeuds de deux graphes n'étant pas définie.
	\item Les poids sur les arêtes doivent tous être positifs ou nul. En effet, l'algorithme de Dijkstra est efficace car, d'itération en itération, il ne doit pas reconsidérer les noeuds pour lesquels il a déjà trouvé la longueur du plus cours chemin par rapport au noeuds initial fourni en argument. Or, si des arêtes de poids négatifs se trouvent dans le graphe, il se peut que la plus petite longueur nécessite d'emprunter des arêtes de plus grand poids pour n'utiliser qu'ensuite une arête négative. Cette mise à jour n'est pas effectuée par l'algorithme, car celle-ci prendrait un temps exponentiel, le nombre de chemin d'un noeud à un autre dans un graphe évoluant exponentiellement avec le nombre de noeuds et d'arêtes.
\end{itemize}

Ayant défini ces quelques conditions, le problème peut maintenant être étudié de façon plus approfondie. Nous pouvons d'or et déjà affirmer que nous aurons besoin d'un outil, d'une fonction, pour déterminer le poids minimal d'une arête entre deux noeuds adjacents. De plus, nous devrons pouvoir considérer les noeuds comme des entités à partir desquelles nous pourrons étudier les relations d'adjacence. Nous développerons davantage ces dernières nécessités en section \ref{sec:th_prob}. Notons enfin que l'algorithme de Dijkstra est aussi d'application sur les graphes dirigés.

\section{Théorie du problème}
\label{sec:th_prob}
%th_prob_section
\subsection{Propriété mathématiques et invariants}
\label{subsec:theorie_prob_prop_math_invariant}
%theorie_prob_prop_math_invariant

Ayant introduit notre problème, tentons de mettre en évidence des propriétés mathématiques nous permettant de trouver un invariant pour générer un algorithme. Nous allons montrer les deux propriétés suivantes : 
\begin{itemize}
	\item Soit une chaîne de longueur minimale de $u_0$ à $v$, dont tous les noeuds intermédiaires ont une distance à $u_0$ plus petite que $v$.
	\item Soit un ensemble de noeuds $S$ et $V$ l'ensemble de tous les noeuds du graphe. Pour chaque $v \in (V \textbackslash S)$, soit $\ell(v)$ la longueur du chemin le plus court de $u_0$ à $v$ dont tous les noeuds intermédiaires sont dans $S$. Alors, pour le noeud $v_x \in (V \textbackslash S)$ tel que $\ell(v_x) \leq \ell(v)$ pour tout $v \in (V \textbackslash S)$, la longueur du plus court chemin entre $u_0$ et $v$ est $\ell(v_x)$.
\end{itemize}

\paragraph{}

Démontrons la première propriété. Soit un noeud intermédiaire $x$. Etant donné que les arêtes sont toutes de poids positifs ou nuls, la longueur du chemin $x-v$ sera positive ou nulle. Donc, la longueur du chemin $u-x-v$ sera plus grande ou égale à la longueur du chemin $u-x$. Comme ceci est valable pour tout noeud intermédiaire $x$, cela démontre la propriété.

\paragraph{}

Démontrons à présent la deuxième propriété. Supposons par l'absurde qu'il existe une longueur $\ell(v_{x_2})$ telle que $\ell(v_{x_2})<\ell(v_x)$. Par hypothèse, on sait que $\ell(v_x)$ est la longueur du chemin le plus court de $u_0$ à $v_x$ dont tous les noeuds intermédiaires sont dans $S$. Donc, $\ell(v_{x_2})$ est la longueur d'un chemin utilisant au moins un noeud hors de $S$. Notons par $v_y$ le premier noeud hors de $S$ rencontré dans le chemin partant de $u_0$ vers $v_x$, de longueur $\ell(v_{x_2})$. Ce chemin sera du type $u_0 - v_y - v_x$. La plus petite longueur possible pour le chemin $u_0 - v_y$ est $\ell(v_y)$, par définition étant donné que ce chemin n'utilise que des noeuds de $S$. Comme les arêtes sont de poids positifs ou nuls, nous savons que le plus court chemin de $v_y$ à $v_x$ a une longueur positive ou nulle : $\ell(v_y - v_x) \geq 0$. Or, par définition de $\ell(v_x)$, nous savons que $\ell(v_x) \leq \ell(v_y)$. Donc, comme $\ell(v_y - v_x) \geq 0$, $\ell(v_x) \leq (\ell(v_y)+\ell(v_y - v_x))=\ell(v_{x_2})$. Ceci contredit notre hypothèse, ce qui prouve la propriété.

\paragraph{}

Sur base de ces deux invariants, nous pouvons construire l'algorithme de Dijkstra. Celui-ci va construire l'ensemble $S$ de la deuxième proposition, en y ajoutant progressivement les noeuds pour lesquels nous avons trouvé la longueur des plus courts chemins depuis $u_0$. 




\subsection{L'algorithme}
\label{subsec:theorie_prob_algo}
% theorie_prob_algo

Ayant remarqué quelques conditions nécessaires au bon fonctionnement de l'algorithme et les invariants utiles à sa construction, nous pouvons  présent décrire l'algorithme.
Pour cela, nous énonçons en premier lieu l'algorithme\footnote{On écrit ici la version de l'algorithme où on renvoit toutes les longueurs des plus courts chemins d'un noeud aux autres du graphe.
Pour ne trouver que la longueur du plus court chemin entre deux noeuds, il suffit d'arrêter l'itération lorsque $u'$ vaut le deuxième noeud de la paire formée avec $u_0$} \cite{cours_graphe}.
On introduit les notations suivantes : $v$ désigne un noeud du graphe $G$, $\ell(v)$ désigne la longueur courante, dans les itérations de l'algorithme, entre $v$ et le noeud $u_0$ fourni en argument et $w(u'v)$ désigne la longueur de l'arête de plus faible longueur joingant $u'$ et $w$, ce qui faut $\infty$ si $u'$ et $w$ ne sont pas adjacents  :

\begin{algorithm}[H]
 \KwData{Graphe $G$, $u_0$ un noeud de $G$}
 \KwResult{Longueurs des plus courts chemins entre un noeud $u_0$ et les autres noeuds du graphe}
 initialization : $\ell(u_0)=0$, $\ell(v)=\infty$ pour $v \neq u_0$, $S:=u_0$, $u'=u_0$ \;
 \While{$G\textbackslash S \neq \emptyset$}{
  pour chaque $v \notin S$, $\ell(v)=$min($\ell(v)$,$\ell(u')+w(u'v)$) \% \bsc{mise à jour de $\ell$} \;
	trouver $\vmin \notin S$ tel que $\ell(\vmin) \leq \ell(v)$ pour tout $v \notin S$\;
	$u':=\vmin$\;
	$S=S \cup \{ u' \}$\;
 }
 \caption{Algorithme de Dijkstra}
 \label{algo:dijkstra}
\end{algorithm}

\paragraph{}

Cet algorithme fonctionne donc par des mises à jour successives des longueurs des plus courts chemins liant un noeud $u_0$ aux autres noeuds du graphe.
Dans l'ensemble $S$ se trouvent les noeuds pour lesquels nous avons trouvé la longueur du plus court chemin le liant à $u_0$.
Nous remarquons donc que dans la phase d'initialisation, la longueur du plus court chemin entre $u_0$ et lui-même est mise à $0$ et que celles de $u_0$ aux autres noeuds du graphe est mise à $\infty$\footnote{Même s'il ne s'agit que d'un détail d'implémentation, notons que nous pourrions stocker les distances entre $u_0$ et les autres noeuds du graphe dans un tableau.}.
L'ensemble $S$ contient donc $u_0$, étant donné que comme il n'y a pas d'arête de poids négatif, la longueur du plus court chemin entre deux noeuds ne peut être inférieure à $0$.
Nous posons alors $u'=u_0$, avec $u'$ le noeud ajouté dans l'ensemble $S$ à l'itération précédente de l'algorithme.

\paragraph{}

Nous entrons alors dans une phase récursive de l'algorithme, qui se poursuit tant que tout les noeuds n'ont pas été ajoutés dans l'ensemble $S$, ce qui signifie que l'algorithme se poursuit tant que toutes les distances des plus courts chemins entre $u_0$ et les autres noeuds du graphe n'ont pas été trouvées.
Notons que dans le cas où nous souhaitons utiliser l'algorithme pour trouver la longueur du plus court chemin entre deux noeuds $u_0$ et $u_1$ et non pas entre un noeud $u_0$ et les autres noeuds du graphe, il suffit d'arrêter la récursion quand $u'=u_1$.

\paragraph{}

Tant que la condition d'itération est satisfaite, nous entrons donc dans la boucle.
La première étape est alors de mettre à jour les distances entre les noeuds hors de $S$ et $u_0$.
En effet, à la phase d'initialisation, nous avons posé les distances entre $u_0$ et tous les autres noeuds du graphe à $\infty$, mais la distance entre $u_0$ et ses voisins est plus petite que $\infty$.
Nous mettons donc à jour les longueurs par rapport à $u_0$ pour les noeuds hors de $S$ de la façon suivante : la longueur d'un chemin\footnote{Et non pas du plus court chemin, car le noeud $v$ n'est pas encore dans l'ensemble $S$, donc $\ell(v)$ n'est qu'une approximation de $d(u_0,v)$, cette dernière notation désignant la longueur du plus court chemin entre $u_0$ et $v$.} entre $u_0$ et $v$ est éale au minimum entre la longueur d'un chemin de $u_0$ à $v$, enregistrée à l'itération précédente, et la longueur du plus court chemin de $u_0$ à $u'$\footnote{Nous désignons cette fois-ci le plus court chemin car $u' \in S$.} additionnée par la longueur de l'arête de plus faible poids liant $u'$ à $v$.
Ainsi, au premier passage dans la boucle, les longueurs $\ell(v)$ seront mises à jour pour les noeuds voisins de $u_0$.
Remarquons d'or et déjà que l'algorithme ne remet à jour les distances que pour les noeuds pour lesquels il n'a pas encore trouvé la longueur du plus court chemin les liant à $u_0$, et non pas pour tous les noeuds du graphe.
Donc, au fur et à mesure des itérations, quand l'ensemble $S$ grandit, cette étape prend de moins en moins de temps, ce qui détermine en grande partie l'efficacité de l'algorithme.
En outre, cette étape est correcte : il n'est pas utile de reconsidérer les longueurs pour tous les noeuds du graphe étant donné qu'il n'y a pas d'arête de poids négatifs.
Nous expliciterons davantage cette propriété par la suite.

\paragraph{}

Une fois les distances mises à jour, nous trouvons alors le noeud hors de $S$ dont la longueur du chemin le liant à $u_0$ est plus petite ue pour tous les autres noeuds de $S$.
Nous avons alors trouvé la longueur du plus court chemin le liant à $u_0$, le rajoutons dans l'ensemble $S$ et le désignons par $u'$ dans la prochaine itération de l'algorithme.
Pourquoi est-il correct de rajouter ce noeud dans l'ensemble $S$ ? Nous allons expliquer ce résultat par une preuve pas récurrence.
Nous allons prouver par récurrence sur les itérations de l'algorithme que, une fois les distances mises à jour, $\ell(\vmin)$ désigne bien la longueur du plus court chemin liant $u_0$ à $\vmin$ et que pour $\ell(v)$, pour tout noeud $v$ du graphe, désigne la longueur du plus court chemin entre $u_0$ et $v$ en utilisant seulement des noeuds de $S$.

\paragraph{}

Considérons d'abord la première itération de l'algorithme (cas de base) : $\vmin$ désignera le voisin de $u_0$ relié à celui-ci par l'arête sortant de $u_0$ de plus faible poids, étant donné que l'étape de mise à jour des longueurs n'a d'effet que pour les noeuds voisins de $u_0$, la distance entre deux noeuds non-adjacents valant $\infty$.
Notons le poids de cette arête par $x$ positif ou nul et cette arête $A$.
La longueur du plus court chemin entre $u_0$ et $\vmin$ sera donc au plus $x$.
Comme toutes les autres arêtes liant $u_0$ à ses voisins sont de poids au moins équivalent et qu'il n'y a pas d'arêtes de poids négatifs, alors un autre chemin vers $\vmin$ ne peut qu'avoir une longueur supérieure ou égale à $x$.
En effet, supposons d'abord que toutes les autres arêtes reliées à $u_0$ soit de poids strictement supérieure à $x$.
Alors, en empruntant un chemin reliant $u_0$ à $\vmin$ en partant avec une autre arête que $A$, nous empruntons un poids strictement supérieure à $x$ et donc pas le plus court chemin car la longueur du plus court chemin reliant $u_0$  $\vmin$ est au plus de $x$\footnote{S'il y avait eu des poids négatifs dans le graphe, alors nous aurions pu ``rattraper'' ce poids supérieur à $x$ en empruntant une arête de poids négative par la suite.
Mais ceci est impossible étant donné que nous excluons les arêtes de poids négatif.}.
Supposons maintenant qu'il existe d'autres arêtes de poids $x$ entre $u_0$ et ses voisins.
Chacune de celles-ci conduit à un autre noeud $v_i$, relié à $\vmin$\footnote{Ces noeuds $v_i$ sont toujours reliés à $\vmin$ dans un graphe non-dirigé : il suffit d'emprunter l'arête dans l'autre sens pour revenir à $u_0$ et puis d'emprunter l'arête $A$.
Dans le cas d'un graphe dirigé, il peut ne pas exister de chemin entre $v_i$ et $\vmin$.}.
Etant donné que toutes les arêtes sont de poids positif ou nul, la longueur du plus court chemin entre $v_i$ et $\vmin$ est positive ou nulle.
Désignons la par $\ell_i$.
Ayant déjà emprunté une arête de poids $x$, si la $\ell_i > 0$, alors le plus court chemin liant $u_0$ à $\vmin$ sans emprunter l'arête $A$ en premier lieu est de poids strictement supérieur à $x$, ce qui n'est pas le plus court chemin.
Si $\ell_i = 0$, alors la longueur du plus court chemin liant $u_0$ à $\vmin$ sans emprunter l'arête $A$ en premier lieu est aussi de $x$.
Comme ceci est valable pour tous les $i$ désignant un voisin $v_i$ de $u_0$ relié à celui-ci par une arête de poids $x$, nous savons alors qu'il n'existe pas de chemin de chemin entre $u_0$ et $\vmin$ de poids inférieure à $x$.
Nous avons donc montré que $x$ est la longueur du plus court chemin reliant $\vmin$ à $u_0$ et donc ce chemin n'est pas unique : il suffit de considérer l'exemple où $u_0$ est relié à $\vmin$ par une arête de poids $x$ positif ou nul, à un autre noeud $v_2$ par une arête de poids $x$ et où $\vmin$ et $v_2$ sont réliés par une arête de poids nul.
Les plus chemins de $u_0$ à $\vmin$ sont donc : $u_0 - \vmin$ et $u_0 - v_2 - \vmin$, ceux-ci étant de poids égaux.

\paragraph{}

Enfin, montrons que $\ell(v)$, pour tout $v \in G$, désigne bien la longueur du plus court chemin entre $u_0$ et $v$ en utilisant seulement des noeuds de $S$, sachant qu'à la première itération, $S$ ne contient que $u_0$. La distance $\ell(u_0)$ n'est pas mise à jour et, valant 0, est bien égale à la longueur du plus court chemin de $u_0$ à $u_0$. Aussi, la longueur du plus court chemin, en utilisant simplement des noeuds de $S$, entre $u_0$ et un noeud $v$ adjacent à $u_0$ est bien égale à $w(u_0 v)$, le poids de l'arête de plus faible poids entre les deux noeuds. De plus, par l'étape de mise à jour de $\ell$ et le fait que $\ell(v)$ est initialisé à $\infty$ pour $v \neq u_0$, $\ell(v)$vaudra bien $w(u_0 v)$ après la mise à jour. Pour $v$ non-adjacent à $u_0$, la distance entre lui et $u_0$ en utilisant seulement des noeuds de $S$ vaut bien $\infty$. $\ell(v)$ pour $v$ non-adjacent n'est donc pas changé. 

\paragraph{}

Ayant démontré le cas de base, démontrons le cas récursif : le noeud $\vmin$ ajouté dans $S_n$\footnote{Nous désignons l'ensemble $S_n$ comme l'ensemble $S$ à la fin de l'itération $n$, donc après avoir exécuté $S:=S \cup \{u'\}$.} à l'itération $n+1$ est relié à $u_0$ par une longueur minimale $\ell(\vmin)$ et, à la fin de l'itération $n+1$, $\ell(v)$, pour tout $v \in G$, désigne la longueur du plus court chemin entre $u_0$ et $v$ en utilisant seulement des noeuds de $S_n$. Nous disposons pour cela de l'hypothèse de récurrence : $\ell(u')$ désigne la longueur du plus court chemin entre $u_0$ et $u'$ et $\ell(v)$, pour tout $v \in G$, défini à la fin de l'étape $n$,  désigne la longueur du plus court chemin entre $u_0$ et $v$ en utilisant seulement des noeuds de $S_n \textbackslash \{u'\}$.  

\paragraph{}

Montrons d'abord que $\ell(\vmin)$ désigne la longueur du plus court chemin entre $u_0$ et $\vmin$. Supposons par l'absurde qu'il existe une distance $\ell_x$ telle que $\ell_x < \ell(\vmin)$. Par l'hypothèse de récurrence, $\ell(\vmin)$ est la longueur du plus court chemin entre $u_0$ et $\vmin$ en utilisant simplement des noeuds de $S_n \textbackslash \{u'\}$. Donc, $\ell_x$ utilise des noeuds hors de $S$ pour aller jusque $\vmin$. Comme les arêtes sont de poids positifs ou nuls, si $\ell_x < \ell(\vmin)$, il existe un noeud $v_y \notin S$ tels que, pour au moins un $v_i$, $\ell(v_y) < \ell(v_i)$, en désignant par $v_i$ un noeud de $S$ emprunté par le chemin de $u_0$ à $\vmin$, de longueur $\ell(\vmin)$, n'empruntant que des noeuds de $S$. Or, ceci est impossible car si cela avait été le cas, nous aurions ajouté $v_y$ dans $S$ à une étape précédente de l'algorithme, par la ligne \textit{trouver $\vmin \notin S$ tel que $\ell(\vmin) \leq \ell(v)$ pour tout $v \notin S$}. Dès lors, $\ell(\vmin)$ désigne la longueur du plus court chemin entre $u_0$ et $\vmin$.

\paragraph{}

Pour ce qui est de la deuxième partie de la propriété, notons par $\ell_{n}(v)$ la longueur du plus court chemin entre $v$ et $u_0$ en utilisant seulement des noeuds de $S_n \textbackslash \{u'\}$, avec $u'$ le noeud ajouté à l'itération $n$ et par $\ell_{n+1}(v)$ la longueur du plus court chemin entre $v$ et $u_0$ en utilisant seulement des noeuds de $S_n$. Le seul noeud susceptible faire diminuer $\ell_{n}(v)$ est celui que nous ajoutons dans $S$, à savoir $u'$, par définition de $\ell_{n}(v)$. Or, la ligne \bsc{mise à jour e $\ell$} exploite justement ce nouveau noeud récemment ajouté pour mettre à jour $\ell_{n}(v)$ en $\ell_{n+1}(v)$, vu $l(u')$ est la longueur du plus court chemin entre $u_0$ et $u'$ et n'utilise de plus que des noeuds de $S_n$, par l'hypothèse de récurrence. Ceci démontre donc la propriété.

\paragraph{}

Au cours de la démonstration de l'algorithme, nous avons donc pu identifié les deux invariants suivants : 
\begin{itemize}
	\item pour $v \in S$, $\ell(v)=d(u_0,v)$, $d(u_0,v)$ désignant la longueur du plus court chemin entre $u_0$ et $v$. De plus, le plus court chemin de $u_0$ à $v$ reste dans $S$.
	\item pour $v \notin S$, $\ell(v) \geq d(u_0,v)$ et $\ell(v)$ est la longueur du plus court chemin de $u_0$ vers $v$ dont tous les noeuds internes sont dans $S$.
\end{itemize}

\paragraph{}

Signalons une fois encore que si le graphe avait comporté des arêtes négatives, les invariants n'auraient pas pu être satisfait. En effet, ce n'est pas parce qu'un noeud $v$ est dans $S$ que $\ell(v)$ désigne la longueur du plus court chemin de $u_0$ à $v$. En effet, considérons le graphe simple complet à trois noeuds $x$, $y$ et $z$. Soit l'arête $x-y$ de poids $3$, l'arête $x-z$ de poids $4$ et l'arête $y-z$ de poids $-2$. Soit $u_0=x$. A sa première itération, comme $3 < 4$, l'algorithme va ajouter $y$ dans $S$ et considérer $\ell(y)=3$. Ensuite, la mise à jour de $\ell(z)$, qui valait $4$ à la précédente itération, donne : $\ell(z)=min(\ell(z),\ell(y)+w(y,z))=min(4,3+(-2))=1$. L'algorithme s'arrêtera donc en posant $\ell(z)=1$, sans remettre à jour $\ell(y)$ à $2$. Sa solution est donc fausse. Pour arriver à un bon résultat, il devrait, à chaque étape, reconsidérer tous les chemins possibles dans S menant à chaque noeud, ce qui a une complexité exponentielle.
 



\section{Spécifications du programmes}
% specification
\subsection{Information de haut niveau}
% info_haut_niveau
\paragraph{Description}
Le programme calcule la distance entre 2 noeuds d'un graphe orienté comportant uniquement des arêtes de poids positif ou nul.

\paragraph{Entrée}
\begin{itemize}
  \item
    La liste de tous les noeuds ainsi que toutes les arêtes du graphe.
  \item
    Le noeud de départ $u_0$.
  \item
    Le noeud d'arrivée $u_1$.
\end{itemize}

\paragraph{Sortie}
Parmis toutes les chaînes reliant $u_0$ à $u_1$, certaines on une taille plus petite ou égale à toutes les autres.
Cette taille est appelée la longueur entre $u_0$ et $u_1$ et est retournée par cet algorithme.

\subsection{Conventions de représentations}
\label{subsec:convention_representation}
%convention_representation : représentation informatique des données et résultats du programme
\paragraph{Conventions de représentation en entrée}
Posons $n$ le nombre de noeuds et $m$ le nombre d'arêtes et $u_i$ pour $i = 1, \ldots, n$ le $i$\ieme{} noeud.

Nous travaillons dans le cas général où le graphe n'est pas spécialement simple,
on ne peut donc représenter le graphe par une matrice d'adjacence, surtout que ça ne serait pas efficace pour un graphe sparse car la complexité spaciale d'une telle matrice est $\bigoh(n^2)$.

L'unique opération qu'on a besoin de faire sur le graphe est de demander la liste des arêtes partant d'un noeud.
Un graphe sous la forme d'une Adjacency List parait donc approprié car la complexité de cette opération pour un noeud $v$ serait alors $\bigoh(\deg(v))$ qui est imbatable car c'est aussi la taille de la sortie.

Si on a en entrée un Edge List, on peut passer en Adjacency List en $\bigoh(m)$ ce qui est acceptable car la complexité de l'algorithme est $\bigoh(m\log(n))$ mais pas vraimen désirable.

Demandons donc une Adjacency List \lstinline|Edge graph[][]|
où \lstinline|graph[i]| donne les arêtes partant du $i$\ieme{} noeud.
Représentons d'ailleurs les noeuds de façon neutre avec $u_i = i-1$.
On aura besoin pour chaque arête de connaitre le poids $w$ et le noeud d'arrivée $v$.
Seulement, pour rester dans la représentation générale des graphes,
définissons une arête comme suit
\begin{lstlisting}
class Edge {
  int u; % source
  int v; % destination
  int w; % weight
}
\end{lstlisting}
Le poids est ici également représenté par un \lstinline|int|.
On va considérer que la somme des poids des arêtes du graphe ne dépasse pas $2^{31}-1$.
On considère aussi qu'il y a moins de $2^{31}-1$ noeuds même si pour un graphe fort sparse, l'algorithme peut terminer assez rapidement même avec un graphe de $2^{32}$ noeuds.

Cette discussion peut être évitée en considérant ici \lstinline|int| comme une indication que ce sont des données qui peuvent être représentée de façon entière sans le lier spécifiquement au \lstinline|int| de Java.

Le noeud de départ et d'arrivée sont simplement spécifier en entier
également avec \lstinline|int u_0| pour le noeud de départ
et \lstinline|int u_1| pour le noeud d'arrivée.

\paragraph{Conventions de représentation en sortie}
Il y aura en sortie un unique entier représentant la longueure minimale que prend une chaîne de $u_0$ à $u_1$.

\paragraph{Conventions de représentation durant le programme}
Du au fait que $u_i = i-1$, $\ell(u_i)$ peut être trivialement représentée par un tableau d'entier \lstinline|int[] l| de taille $n$.
Une question se pose cependant sur la représentation de l'infini.
Comme toutes les longueurs sont positives, on peut utiliser une valeur sentinelle comme $-1$ sans poser d'ambiguïter.
On aurait donc
$$\lstinline|l[i]| =
\begin{cases}
  -1 & \text{si } \ell(u_{i+1}) = \infty\\
  \ell(u_{i+1}) & \text{sinon}.
\end{cases}$$

Vient maintenant la question de la structure de donnée nous permettant d'obtenir $\vmin$ dans l'algorithme~\ref{algo:dijkstra}.
Notre Adjacency List nous permet de faire la ligne 3 en $\bigoh(\degout(u') f(n))$ où $f(n)$ est la complexité de mettre à jour $\ell(v)$.
La ligne 4 a une complexité de $\bigoh(g(n))$ où $g(n)$ est la complexité de trouver $\vmin$.
La complexité de l'algorithme est donc $\bigoh(mf(n) + ng(n))$.
Comparons à présent plusieurs structures de donnée possibles nous permettant de rechercher $\vmin$
\begin{description}
  \item[Aucune] Avec simplement \lstinline|l|, on a $f(n) = \bigoh(1)$ et $g(n) = \bigoh(n)$ et donc $\bigoh(m + n^2)$ ce qui pour un graphe simple vaut $\bigoh(n^2)$ car $m \leq \frac{n(n-1)}{2}$ pour un graph simple. C'est une bonne complexité pour un graphe dense mais comme on va le voir on sait faire bien mieux pour un graphe un peu sparse.
  \item[Un heap avec location unaware entries] Comme on ne sait pas modifier la clef d'une entrée, lorsqu'on veut mettre à jour $\ell(v)$, il faut réajouter $v$ avec le nouveau $\ell(v)$ et se rappeler d'ignorer
    l'ancien quand il sortira de la priority queue. Ceci peut être fait en calculant $S$ à l'aide d'un Bitset.
    Dans le pire cas où on met à jour à chaque coup, le nombre d'entrée dans le heap peut atteindre $m$, ce qui donne
    $f(n, m) = \bigoh(\log(m))$ et $g(n, m) = \bigoh(\log(m))$ et pour un graphe simple,
    $f(n, m) = \bigoh(2\log(n))$ et $g(n, m) = \bigoh(2\log(n))$ ce qui fait aussi du $\bigoh(\log(n))$ mais avec une moins bonne constante.
    Ça rajoute aussi du $m \log(m)$ pour toutes les $m$ boucles inutiles dans le pire cas où on tirera une entrée à ignorer.
  \item[Un heap avec location aware entries] On peut alors modifier directement l'entrée correspondante à $v$ lorsqu'on met
    à jour $\ell(v)$ ce qui va simplement replacer $(v, \ell(v))$ dans l'arbre et qui est d'ailleurs bien plus léger qu'un ajout.
    On a alors simplement $f(n) = \log(n)$ et $g(n) = \log(n)$ ce qui fait une complexité $\bigoh(m\log(n))$ mais avec une constante bien
    inférieure à celle du précédent.
\end{description}
Le heap formant un arbre binaire complet, il peut être stocké de façon très efficace dans un tableau \lstinline|int[] heap|
où $\lstinline|l[heap[i]]| \leq \lstinline|l[heap[2*(i+1)-1]]|$ et $\lstinline|l[heap[i]]| \leq \lstinline|l[heap[2*(i+1)-2]]|$.
Pour la location awareness, il nous suffit d'un autre tableau \lstinline|int[] back|
donnant à chaque noeud sa place dans le tableau donc $\lstinline|back[v]| = \lstinline|i| \Leftrightarrow \lstinline|heap[i]| = \lstinline|v|$.


%\section{Décomposition en sous-problèmes}
%% decomposition_sous_problemes

%\newpage

%%%%%%%%%%%%%%%%%%%%%%%%%%%%%%%%%%%%%%%%%%%%%%%%%%%%%%%%%%%%%
%% BIBLIOGRAPHY AND OTHER LISTS
%%%%%%%%%%%%%%%%%%%%%%%%%%%%%%%%%%%%%%%%%%%%%%%%%%%%%%%%%%%%%
%% A small distance to the other stuff in the table of contents (toc)
%\addtocontents{toc}{\protect\vspace*{\baselineskip}}
\newpage
%% The Bibliography
%% ==> You need a file 'literature.bib' for this.
%% ==> You need to run BibTeX for this (Project | Properties... | Uses BibTeX)
\addcontentsline{toc}{chapter}{Bibliography} %'Bibliography' into toc
\nocite{*} %Even non-cited BibTeX-Entries will be shown.
\bibliographystyle{ieeetr} %Style of Bibliography: plain / apalike / amsalpha / ...
\bibliography{Biblio} %You need a file 'literature.bib' for this.

%% The List of Figures
%\clearpage
%\addcontentsline{toc}{chapter}{List of Figures}
%\listoffigures

%% The List of Tables
%\clearpage
%\addcontentsline{toc}{chapter}{List of Tables}
%\listoftables


%%%%%%%%%%%%%%%%%%%%%%%%%%%%%%%%%%%%%%%%%%%%%%%%%%%%%%%%%%%%%
%% APPENDICES
%%%%%%%%%%%%%%%%%%%%%%%%%%%%%%%%%%%%%%%%%%%%%%%%%%%%%%%%%%%%%
%\newpage
%\appendix
%% ==> Write your text here or include other files.

%\input{FileName} %You need a file 'FileName.tex' for this.


\end{document}

