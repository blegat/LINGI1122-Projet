%En-t�te classique pour un rapport en EPL
%========================================
%\usepackage{fullpage} %si on veut en page compl�te

\usepackage[french]{babel} %En francais
\usepackage[utf8]{inputenc} %Caracteres accentu�s
\usepackage[T1]{fontenc} 	%Police accentu�e
\usepackage{lmodern}			%Police vectorielle (haute qualit�)
\usepackage{vmargin} %Marges normales en A4
\usepackage{amsmath} %Insertion d'equations
%\usepackage{theorem}
%\usepackage{ntheorem} 
 

\usepackage{amsfonts}
\usepackage{amssymb}
%\usepackage{algorithm}
\usepackage{arydshln}  % pour pointill�s dans matrice
\usepackage{algpseudocode}
%\usepackage{amsfont} % pour $\mathbb{R}$
\usepackage{amssymb}

\usepackage{graphicx} %Images dans le PDF
\usepackage{epstopdf} %importer des eps (� laisser sous package graphicx !
\graphicspath{{Figures/}}
\usepackage{float} 		%Flottants : Figures,Tables
\usepackage{color}		%Utilisation de couleurs
\definecolor{gris}{gray}{0.5}
\usepackage{eurosym}
\usepackage[hyphens]{url}
\usepackage{hyperref}
% \usepackage{breakurl}
\usepackage{wrapfig}
\usepackage{multirow} 
\usepackage{rotating}
\usepackage[small,bf]{caption}
\usepackage{textcomp}
\usepackage{subfigure}
\usepackage{amsthm}
\usepackage{pgf}
\usepackage{mathtools}
\usepackage[final]{pdfpages} 

\newtheorem*{theorem}{Th�or�me} % pour num�roter : \newtheorem{theorem}{Th�or�me}[section]
\newtheorem{mydef}{D�finition}

%========================================
\usepackage{listings}	%Inclusion de code-source
\lstset{							%Param�tres g�n�raux pour les inclusions de code
	flexiblecolumns=true,
	numbers=left,
	numberstyle=\ttfamily\tiny,
	keywordstyle=\textcolor{blue},
	stringstyle=\textcolor{red},
	commentstyle=\textcolor{green},
	breaklines=true,
	extendedchars=true,
	basicstyle=\ttfamily\scriptsize,
	showstringspaces=false,
	captionpos=b
	}
\renewcommand{\lstlistingname}{\textsc{Code}}	%Remplacer 'Listing' par 'Code' dans les l�gendes
\usepackage[french,boxed,linesnumbered]{algorithm2e}	%Package algorithme avec options


\hypersetup{
colorlinks,%
citecolor=black,%
filecolor=black,%
linkcolor=black,%
urlcolor=black
} 
\usepackage[framed,numbered,autolinebreaks,useliterate]{mcode}
%% Define a new 'leo' style for the package that will use a smaller font.
\makeatletter
\def\url@leostyle{%
  \@ifundefined{selectfont}{\def\UrlFont{\sf}}{\def\UrlFont{\small\ttfamily}}}
\makeatother
%% Now actually use the newly defined style.
\urlstyle{leo}


\makeatletter
\renewenvironment{thebibliography}[1]{%
%     \section*{\refname}%
%      \@mkboth{\MakeUppercase\refname}{\MakeUppercase\refname}%
      \list{\@biblabel{\@arabic\c@enumiv}}%
           {\settowidth\labelwidth{\@biblabel{#1}}%
            \leftmargin\labelwidth
            \advance\leftmargin\labelsep
            \@openbib@code
            \usecounter{enumiv}%
            \let\p@enumiv\@empty
            \renewcommand\theenumiv{\@arabic\c@enumiv}}%
      \sloppy
      \clubpenalty4000
      \@clubpenalty \clubpenalty
      \widowpenalty4000%
      \sfcode`\.\@m}
     {\def\@noitemerr
       {\@latex@warning{Empty `thebibliography' environment}}%
      \endlist}
\makeatother

