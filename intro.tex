% introduction

Dans le cadre du cours de \texttt{Méthodes et conceptions de programmes}, il nous a été proposé d'étudier le problème de l'algorithme de Dijkstra. Cet algorithme revêt une importance considérable en théorie des graphes, un domaine qui prend de plus en plus d'ampleur de nos jours. Afin d'introduire la notion de graphe et l'algorithme de Dijkstra, considérons un \bsc{gps} dont la charge est de déterminer la plus court chemin entre deux villes fournies. Nous pouvons modéliser ce problème en représentant les villes comme les noeuds d'un graphe et les arêtes entre les noeuds comme les routes entre les villes. Chacune des arêtes sera pondérée par un nombre naturel\footnote{Dans le cadre de l'algorithme de Dijkstra, nous verrons qu'il est requis de n'avoir que des poids positifs ou nuls sur les arêtes.}. Le poids d'une arête peut donc exprimer l'avantage que l'on a à empruter la route correspondante (les autoroutes se verront attribuées des poids plus petit que les petites routes de campagne, dans le cadre de la recherche du plus court chemin entre deux villes). Ainsi, un chemin d'une ville à l'autre revient à emprunter une suite d'arêtes, chacune reliée à la précédente par un noeud. La longueur d'une telle suite d'arêtes est alors définie comme la somme des poids des arêtes de la suite.

\paragraph{}
Dans le cadre de notre exemple ci-dessus, il s'agit de trouver la plus faible longueur d'un chemin entre deux noeuds du graphe. Ce problème est celui résolu par l'algorithme de Dijkstra, dans le cas où tous les poids des arêtes sont positifs. Dans la suite du document, nous tenterons de spécifier davantage la théorie du problème ainsi que les conventions de représentation que nous nous sommes proposés d'utiliser.