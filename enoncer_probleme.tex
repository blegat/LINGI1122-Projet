%enoncer_probleme

Après avoir introduit notre problème au cours de l'introduction, énonçons le de manière plus formelle : il s'agit de trouver, dans un graphe pondéré donné, la longueur du plus court chemin entre deux noeuds d'une paire donnée, ou bien les longueurs des plus courts chemins entre un noeud donné et tous les autres noeuds du graphe. Nous verrons en section \ref{sec:poser_prob} qu'il est nécessaire que les poids sur les arêtes soient positifs ou nuls. Le programme doit donc retourner un entier représentant la longueur du plus court chemin entre deux noeuds donnés dans un graphe donné ou une structure de donnée contenant les longueurs des plus courts chemins entre un noeud donné dans un graphe donné et les autres noeuds du graphe. La structure de donnée employée sera spécifiée en section \ref{subsec:convention_representation}.